% Chapter Template

\chapter{Conclusion} % Main chapter title

\label{chapter-conclusion} % Change X to a consecutive number; for referencing this chapter elsewhere, use \ref{ChapterX}

\lhead{Chapter 7. \emph{Conclusion}} % Change X to a consecutive number; this is for the header on each page - perhaps a shortened title

Human-Robot interaction is a very complex subject, that involves different problematics. In order to be socially acceptable the robot needs to be able to interact with a human in a simple, efficient, and natural way. In this thesis we presented a framework that allows a robot to perform cooperative tasks with humans. We will now review the main aspects of our system:

\begin{itemize}
\item Our system is able to model and maintain a mental belief model for the robot and for all the present agents, using geometrical reasoning to compute relationships between entities. These ideas are inspired by psychological subjects, like perspective taking.
\item By using belief models, contextual informations, planning with MDPs,  geometrical reasoning, and a BN model, the robot is able to infer humans' actions and intentions. Our algorithm has been compared to users' skills in a user study.
\item The robot has simple mechanisms for dialogue. It is able to receive goals from a tablet application, and it has been interfaced with a speech recognition and a text to speech system.
\item The robot is able to choose and manage goals, starting from information obtained from Situation Assessment.
\item The robot is able to build plans for all the present agents and to take into account their knowledge and expertise. Our system is interfaced with two different planners: an HTN-based human-aware planner, and a novel multi-agent MDP planner.
\item Plans can be explained to users based on their expertise on tasks, avoiding to convey to many useless information. This algorithm has been tested in a user study, where it was well perceived by users.
\item The robot is able to manage plans in a flexible way. We presented several different algorithms, to execute plans in a sequential or parallel way. While managing a plan the robot can execute its part while monitoring humans to evaluate if their actions are consistent with the shared plan.
\item Our robot is able to execute actions in a robust and safe way. The system is able to take into account humans when planning its motions and to stop if one of its actions could endager them.
\item The system is able to execute joint actions with humans. We presented a special framework based on MOMDPs that we used to execute cooperative activities. We presented two of these activities: handover and guiding.
\item The system is able to work in complex scenarios in the real world. We presented two case studies in this thesis. One of these was conducted in the airport of Schiphol, a very complex environment, where it was evaluated positively by users.
\end{itemize}


% First, we presented the situation assessment capacities of our system, which is able to build and maintain the belief models of the robot and of other humans. We presented a novel algorithm to infer human intentions and actions, using BNs and MDPs, and performed a preliminary user-study to compare it to the capacity of humans to infer others' intentions.

% After this, we showed how our system is able to reason on the current status of the world to choose and manage its own goals.

% In the next chapter, we explained how we are able to build, explain, negotiate, and manage plans. We presented two different planners used integrated in our system, an HTN-based planner called HATP, and a novel multi-agent MDP planner, called HAPP. We showed how plans can be generated to take into account users' knowledge and capacities, and how we explain plans to user in a natural way. We also presented different algorithms to flexibly manage plans, by allowing users the freedom to vary from the robots' plan if the robot perceives the user as competent in performing a certian task. Finally, we showed in a user study that our algorithm to explain plans to users, based on their knowledge, is perceived in a good way by them.

% After this, we showed how we execute plans, focusing in particular on a special framework based on MOMDPs that we introduced to execute joint actions.

% Finally, we showed two case studies where we used our system: a robot guide, which was deployed at the airport of Schiphol to guide passengers to their gates, and a domestic helper, able to clean cooperatively a table with a human.



\section{Perspectives}
In this work, we tried to built an architecture that includes several novel and experimental systems. This architecture is just a starting point for further studies, and there are several developments that could be done.

\begin{itemize}
\item Introduce learning algorithms. We believe that learning algorithms could enhance several aspects of our system. For example, the robot could learn the users' habits and capacities, to be more able to understand his behaviors. Similarly, learning algorithms, could enhance the execution of motions of the robot, adapting them to the current scenario and user.
\item More user studies. During the development of this work, we managed to perform several preliminary user studies, of which the most important was the one conducted in the Schipol airport with a robot guide. It would be interesting to perform more in-depth user studies, on several sub-systems and on the architecture as whole, to guide our future developments 
\item Introduce Dialog. Dialog is a very important part of interaction. In our system, the robot is able to perform very simple and not natural dialogue. While natural dialogue is a very complex problem, we believe that if we would introduce more communication capacities in our robot it would be perceived better by users.
\item More complex models. While developing our system we tried to find a balance between the quality of our representations and the overall complexity  of the architecture. Improving the complexity of the model and studying more efficient algorithms could improve the quality of our system.
\end{itemize}
